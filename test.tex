\documentclass[1,DS,p,e,prof]{phBook}
%\documentclass[../coursseconde]{subfiles}
%Options : 2, 1, T, SNT (classes), p, b,prof, exos, cours, DS, noir, a4, a5, twocolumn, landscape
\def\datedevoir{6/01/26} % date du DS à indiquer

\graphicspath{{figures/}}

\begin{document}
\setcounter{chapter}{3} %Numéro du chapitre précédent
\chapter{\prof{CORRECTION}\eleve{NOM : \hfill Devoir
}} %Titre du chapitre
%%%%%%%%%%%%%%%%%%%%%%%%

% sens de variations des fonctions : deux quadratiques, trois fonctions du troisième degré de sorte que pour f'(x) soient avec  \Delta de signes différents, une fraction rationnelle, une de la forme ax+b/x.
\begin{enonce}
  Pour chacune des fonctions suivantes :
  \begin{itemize}
    \item Calculer la dérivée \(f'(x)\).
      % \item Étudier le signe de \(f'(x)\).
    \item Déterminer les sens de variations de la fonction \(f\).
  \end{itemize}
  \begin{enumerate}
    \item \(f(x)=x^2-4x+3\).
    \item \(f(x)=-2x^2+4x+5\).
    \item \(f(x)=-x^3+3x^2-3x+2\).
      % \item \(f(x)=x^3-6x^2+9x+1\).
    \item \(f(x)=-2x^3+3x^2+12x-5\).
    \item \(f(x)=\dfrac{2x+1}{x-2}\) définie sur \(\mathbb{R}\setminus\{2\}\).
    \item \(f(x)=3x+\dfrac{48}{x}\), définie sur \(\mathbb{R}\setminus\{0\}\).
  \end{enumerate}
\end{enonce}

%%%%%%%%%%%%%%%%%
% Fin exercice %%
%%%%%%%%%%%%%%%%%

\begin{exercice}
  \begin{enonce}
    Donner les variations de $f(x)=\frac{2x-1}{x+3}$ sur son domaine de définition.
  \end{enonce}
  %%%%%%%%%%%

\end{exercice}
%%%%%%%%%%%%%%%%%%%%%
%%% Fin exercice %%%%
%%%%%%%%%%%%%%%%%%%%%

\end{document}
